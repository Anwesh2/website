% Adapted from layout by GaelVaroquaux
%  http://gael-varoquaux.info
%

\documentclass{article} %{{{--

\usepackage[paper=a4paper,
	    top=2cm,
	    left=1.35cm,
	    width=18.2cm,
	    bottom=2cm
	    ]{geometry}
            %margin=4cm,

\usepackage{calc}
\usepackage[T1]{fontenc}
\usepackage[utf8]{inputenc}
\usepackage{lmodern}
\usepackage{color,hyperref}
\usepackage{graphicx}
\usepackage{multicol}

\usepackage{wasysym} % For phone symbol
\usepackage{url}

\def\bf{\bfseries}
\def\sf{\sffamily}
\def\sl{\slshape}
% Semi condensed bold

\definecolor{deep_blue}{rgb}{0,.2,.5}
\definecolor{dark_blue}{rgb}{0,.1,.3}
\definecolor{myblue}{rgb}{.01,0.21,0.71}
\definecolor{gray}{rgb}{.5, .5, .5}

\hypersetup{pdftex,  % needed for pdflatex
  breaklinks=true,  % so long urls are correctly broken across lines
  colorlinks=true,
  urlcolor=myblue,
  %linkcolor=darkblue,
  %citecolor=darkgreen,
  }


%% This gives us fun enumeration environments. 
\usepackage{enumitem}

%% More layout: Get rid of indenting throughout entire document
\setlength{\parindent}{0in}

%% Reference the last page in the page number
%
\usepackage{fancyhdr,lastpage}
\pagestyle{fancy}
%\pagestyle{empty}      % Uncomment this to get rid of page numbers
\fancyhf{}\renewcommand{\headrulewidth}{0pt}
%\fancyfootoffset{\marginparsep}+\marginparwidth}
\lfoot{
  \hspace{-2\marginparsep}
  \,\hfill \arabic{page} of \protect\pageref*{LastPage}  \hfill \,\\
  \,\hfill {\footnotesize \textcolor{gray}{updated January 2014~~~~~}} \hfill\,
}

\newcommand{\mydate}[1]{{\textcolor{gray}{\footnotesize #1}}}


\newcommand{\makeheading}[1]%
        {%\hspace*{-\marginparsep minus \marginparwidth}%
         %\begin{minipage}[t]{\textwidth+\marginparwidth+\marginparsep}%
         \begin{minipage}[t]{\textwidth}%
                {\Large #1}\\%[-0.5\baselineskip]%
                \vskip 0.2\baselineskip
                 \color{deep_blue}{\rule{\columnwidth}{3pt}}%
         \end{minipage}
	 \vskip 1.\baselineskip plus 2\baselineskip minus 1.\baselineskip
	}

\newlength\sidebarwidth
\setlength\sidebarwidth{3.6cm}

\newcommand{\topic}[3][]%
	 {\pagebreak[2]%
	 \vskip 1.5\baselineskip plus 3\baselineskip minus 0.7\baselineskip
	 \begin{minipage}{\textwidth}
         \phantomsection\addcontentsline{toc}{section}{#1}%
         \nopagebreak\hspace{0in}%
         \nopagebreak\begin{minipage}[t]{\sidebarwidth - .2cm}
         \raggedleft \bf\sf 
	 \color{deep_blue}{\Large #2}
	 \end{minipage}%
	 \hfill
	 \begin{minipage}[t]{\linewidth - \sidebarwidth}
	 \nopagebreak{\color{deep_blue}%
		    \rule{0pt}{\baselineskip}%
		    \rule{\linewidth}{2.5pt}%
		    \llap{\raisebox{.3\baselineskip}{\sf #1}}%
		    \vspace*{.1\baselineskip}%
		    }%
	 #3%
	 \end{minipage}
	 \end{minipage}}

\newcommand{\smalltopic}[2]%
	 {\pagebreak[2]%
	 \vskip 1\baselineskip plus 2\baselineskip minus 0.3\baselineskip
	 \begin{minipage}{\textwidth}
	 %\hspace{-\marginparsep minus \marginparwidth}%
         \phantomsection\addcontentsline{toc}{subsection}{#1}%
         \nopagebreak\hspace{0in}%
         \nopagebreak\begin{minipage}[t]{\sidebarwidth - .2cm}
         \raggedleft \bf\sf %\vskip -0.5\baselineskip
	 \textcolor{dark_blue}{\large #1}%
	 \end{minipage}%
	 \hfill
	 \begin{minipage}[t]{\linewidth - \sidebarwidth}
	 \nopagebreak{%
	    %\vspace{-.7\baselineskip}%
	    \rule{\linewidth}{.5pt}%
	    \vspace{.1\baselineskip}%
	    }%
	    #2
	 \end{minipage}
	 \end{minipage}}

\newcommand{\subtopic}[3][]
	 {\begin{minipage}{\textwidth}
	 \vspace*{.4\baselineskip}
         \nopagebreak\hspace{0in}%
         \nopagebreak\begin{minipage}[t]{\sidebarwidth - .2cm}
	 % Super posh: using semi-bold condensed fonts. Works only with
	 % lmodern
         \raggedleft {\sf\fontseries{sbc}\selectfont #2}
	 %{\small\sl\\[-0.2\baselineskip] #1}
         {\\[-0.2\baselineskip] \textcolor{gray}{\footnotesize #1}}
	 \end{minipage}%
	 \hfill
	 \begin{minipage}[t]{\linewidth - \sidebarwidth}
	 #3%
	 \end{minipage}%
	 \vspace*{.2\baselineskip plus 1\baselineskip minus
	 .2\baselineskip}%
	 \end{minipage}}

\newcommand{\dateonly}[2][]
	 {\begin{minipage}{\textwidth}
	 \vspace*{.4\baselineskip}
         \nopagebreak\hspace{0in}%
         \nopagebreak\begin{minipage}[t]{\sidebarwidth - .2cm}
         \raggedleft {~}
         {\\[-\baselineskip] \textcolor{gray}{\footnotesize #1}}
	 \end{minipage}%
	 \hfill
	 \begin{minipage}[t]{\linewidth - \sidebarwidth}
	 #2%
	 \end{minipage}%
	 \vspace*{.2\baselineskip plus 1\baselineskip minus
	 .2\baselineskip}%
	 \end{minipage}}

\newcommand{\sidenote}[2]
	 {\vspace*{-.2\baselineskip}\begin{minipage}{\textwidth}
         \nopagebreak\hspace{0in}%
         \nopagebreak\begin{minipage}[t]{\sidebarwidth - .2cm}
         \raggedleft {#1}
	 \end{minipage}%
	 \hfill
	 \begin{minipage}[t]{\linewidth - \sidebarwidth}
	 #2%
	 \end{minipage}%
	 \vspace*{.5\baselineskip plus 1\baselineskip minus
	 .2\baselineskip}%
	 \end{minipage}}

% New lists environments 
\newlist{outerlist}{itemize}{1}
\setlist[outerlist]{font=\sffamily\bfseries, label=\textbullet}
\setitemize{topsep=0ex, partopsep=0ex}
\setdescription{font=\normalfont\sffamily\bfseries, itemsep=.5ex,
    parsep=.5ex, leftmargin=3ex}

\newcommand{\blankline}{\quad\pagebreak[2]}

\def\mydot{\textcolor{deep_blue}{\rule{1ex}{1ex}}}

%%%%%%%%%%%%%%%%%%%%%%%%%%%%%%%%%%%%%%%%%%%%%%%%%%%%%%%%%%%%%%%%%%%%%--}}}%
\begin{document}
\makeheading{
\begin{minipage}[B]{0.5\textwidth}
    %\vfill
    \vspace*{-.5\baselineskip}%
    \parbox{10cm}{
	\hskip -0.1cm
	{\Huge\bf\sf \color{deep_blue} J\huge \hskip -0.05cm AKE %
	 \Huge \color{deep_blue} V\huge \hskip -0.05cm ANDERPLAS}
	\\[-.1\baselineskip]
	{\bf\sf Director of Research -- Physical Sciences}
        \\
        eScience Institute, University of Washington
    }
\end{minipage}
\hfill
\begin{minipage}[B]{8cm}
    \raggedleft
    \,\vskip -1em
    \small
        Paul G. Allen Center\\
        Box 352350\\
        Seattle WA 98195-2350\\
	{\texttt {jakevdp@cs.washington.edu}}%
    \vspace*{-.5\baselineskip}%
\end{minipage}
}

%\begin{center} 
%\begin{minipage}{15cm}
\begin{multicols}{2}
\sloppy

%\textcolor{deep_blue}{\bf\sf Research interests}:
%Cosmology, weak lensing, data mining and automated learning algorithms for
%large astronomical data sets.

\begin{itemize}[leftmargin=2ex, itemsep=0ex]
\item[\mydot]
I am part of the University of Washington's eScience institute, an
interdisciplinary program designed to support data-driven discovery in a
wide range of scientific fields.  Alongside my institute duties, I work on
my own research in astronomy, astrophysics, machine learning, and
scalable computation.

\item[\mydot]
My research interests are in astronomy, astrophysics, machine learning, and
scalable computation. Previously I was an an NSF post-doctoral fellow,
working jointly between the Astronomy and Computer Science departments
at the University of Washington. During that fellowship, I began work on
pushing the limits of large, distributed array-based computation with
{\it SciDB}, and develop techniques for data mining and machine learning
in large astronomical surveys like {\it LSST}.

\item[\mydot]
My PhD research focused on {\it weak lensing}, a technique which utilizes small
gravitational perterbations of light paths to learn about the distribution
of matter in the universe.

\item[\mydot]
I am interested in encouraging reproducible and open research practices
across scientific disciplines.  To this end I spend significant time
developing and presenting tutorials on scientific computing in the Python
programming language, and investing time developing a variety of open-source
scientific computing tools.

\end{itemize}
\end{multicols}
\vspace*{-1.5em}
\fussy
%\end{minipage}
%\end{center}

%%%%%%%%%%%%%%%%%%%%%%%%%%%%%%%%%%%%%%%%%%%%%%%%%%%%%%%%%%%%%%%%%%%%%%%%%%%
\topic{E \large\hskip -1ex DUCATION}{~}

    \subtopic[2006-2012]{\bf PhD}{
        University of Washington, Seattle, WA, advised by Andrew Connolly
        and Bhuvnesh Jain\\
	Thesis: \href{http://adsabs.harvard.edu/abs/2013arXiv1301.6657V}{
          Karhunen-Loeve Analysis for Weak Gravitational Lensing}
    }

    \subtopic[2006-2007]{\bf MS}{
      University of Washington, Seattle, WA, advised by Craig Hogan
      and Andrew Becker\\
    }

    \subtopic[1999-2003]{\bf BS}{
      Calvin College, Grand Rapids MI\\
      Major: Physics; Minors: Mathematics \& Japanese\\
      4-year letterman \& 1-year captain of the varsity swim team
    }

%%%%%%%%%%%%%%%%%%%%%%%%%%%%%%%%%%%%%%%%%%%%%%%%%%%%%%%%%%%%%%%%%%%%%%%%%%%
\topic{E \large\hskip -1ex XPERIENCE}{~}

\vspace*{-0.5\baselineskip}
\smalltopic{Employment}{}

    \subtopic[2014--Present]{UW eScience}{
      Director of Research, Physical Sciences.\\
      eScience Institute, University of Washington.}

    \subtopic[2013--2014]{UW Computer Science}{
      NSF post-doctoral fellowship, CI-TraCS program.\\
      Department of Computer Science, University of Washington.
      Supervised by Magda Balazinska}

    \subtopic[2012--2013]{UW Astronomy}{
      Postdoctoral Researcher, LSST Image Simulation group.\\
      Department of Astronomy, University of Washington.
      Supervised by Andrew Connolly
    }

    \subtopic[2010--2012]{UW Planetarium}{
      WorldWide Telescope Planetarium Project Coordinator\\
      University of Washington Planetarium, Seattle WA \& Microsoft Research, Redmond WA
    }

    \subtopic[2008--2010]{UW Planetarium}{
      K-12 and Community Outreach Coordinator\\
      University of Washington Planetarium, Seattle WA
    }

    \subtopic[2004--2006]{Mount Hermon}{
      Experiential Science Educator (4th-8th grade students)\\
      Mount Hermon Outdoor Science School, Santa Cruz CA
    }

    \subtopic[2004--2005]{Summit Adventure}{
      Wilderness Instructor: Backpacking, Rock Climbing, and Mountaineering\\
      Summit Adventure, Bass Lake CA
    }

    \subtopic[2003--2004]{Japan ESL}{
      Teacher and Tutor of English as a second language\\
      Sendai Gakusei Sentaa, Sendai, Japan
    }

\smalltopic{Volunteering}{}

    \subtopic[2013--Present]{Neighborhood Advocacy}{
      As co-chair of the North Delridge Neighborhood Council, I facilitate
      community gatherings, service work, and other advocacy
      in our mixed-income neighborhood in southwest Seattle.
    }

    \subtopic[2010--2013]{Safe Streets Advocacy}{
      As a founder of West Seattle Greenways and transportation chair of
      the Delridge Neighborhood Council, I led the effort to secure grants
      and city funding totaling over \$2 million for pedestrian
      and bicycle safety improvements in the neighborhood.
    }

    \subtopic[2009--2013]{Pacific Science Center}{
      As a Science Communication Fellow, I facilitate activities for
      museum visitors and give occasional community talks on astronomy
      and astrophysics.
    }

    \subtopic[2007--2012]{Sierra Club}{
      As a program leader for the Sierra Club's {\it Inner City Outings}
      program, I led 3-4 hiking \& camping trips each year with inner-city
      youth.
    }

    \subtopic[2006--2012]{UW Planetarium}{
      Through my graduate career, I participated in the University of
      Washington Planetarium's K-12
      outreach program, facilitating planetarium shows several times each
      quarter for visitors aged 4 to adult.
    }

\smalltopic{Formal Teaching}{}

    \subtopic[Fall 2013]{Astr 599}{
      Scientific computing for Astronomy\\
      {\it lecturer -- University of Washington}}

    \subtopic[Fall 2008]{Astr 102}{
      Introductory Astronomy for Science Majors\\
      {\it head teaching assistant -- University of Washington}}

    \subtopic[Fall 2007]{Astr 101}{
      Introductory Astronomy\\
      {\it teaching assistant -- University of Washington}}

\smalltopic{Students Mentored}{}

    \subtopic[2012--2013]{SungWon Kwak}{
      Undergraduate, University of Washington Astronomy\\
      {\it Superimposed High Redshift Spectra}
    }

    \subtopic[2008--2009]{Andy Barr \&\\Devon McMinn}{
      Undergraduates, University of Washington Pre-MAP program\\
      {\it Astronomical Data Processing with LLE}
    }


%%%%%%%%%%%%%%%%%%%%%%%%%%%%%%%%%%%%%%%%%%%%%%%%%%%%%%%%%%%%%%%%%%%%%%%%%%%
\topic{A \large\hskip -1ex WARDS \& HONORS}{~}

    \subtopic[July 2013]{Data Visualization}{
      Runner-up in the 2013 {\it John Hunter Excellence in Plotting Competition}
    }

    \subtopic[October 2012]{CIDU Best Paper}{
      Recipient of the Best Paper Award, 2012 Conference on Intelligent
      Data Understanding (CIDU).
    }

    \subtopic[Spring 2012]{NSF Fellowship}{
      Recipient of a 3-year NSF prize fellowship through the
      office of CyberInfrastructure CI-TraCS program.
      NSF Award \#1226371.
    }

    \subtopic[1999-2003]{Calvin College}{
      4-year recipient of the Calvin College Presidential Scholarship.
    }

    \subtopic[2000-2001]{Calvin College}{
      Recipient of the Roger D. Griffioen Scholarship for Physics majors.
    }

\pagebreak
%%%%%%%%%%%%%%%%%%%%%%%%%%%%%%%%%%%%%%%%%%%%%%%%%%%%%%%%%%%%%%%%%%%%%%%%%%%
\topic{C \large\hskip -1ex OMPUTING}{
  I am an active developer, maintainer, and contributor to several
  well-known scientific computing packages in the Python community.
  See my github profile
  (\href{http://github.com/jakevdp}{http://github.com/jakevdp})
  for details.}

%\vspace*{-0.5\baselineskip}
\smalltopic{Skills}{

  \begin{itemize}[leftmargin=0ex, itemsep=0ex, labelindent=-2ex, parsep=.5ex]
    \item[\mydot] Proficient open source developer, with a specialization in
      scientific computing, including visualization, data mining and machine
      learning.
    \item[\mydot] Expert in the Python Language and extensions such as Cython;
      very good knowledge of C, C++, and interfacing to legacy Fortran code.
    \item[\mydot] Experience with a variety of tools and languages, including
      bash, csh, \LaTeX{}, HTML, Javascript, Git, various database query
      languages, web templating enginces such as Jinja, etc.
    \item[\mydot] Author of {\it Pythonic Perambulations}, a popular Python
      blog covering scientific computing, visualization, and whimsical
      distractions: \href{http://jakevdp.github.io}{http://jakevdp.github.io}
  \end{itemize}

}

\smalltopic{Software}{}

   \subtopic[2010--Present]{Scikit-Learn}{
     I a member of the core team of
     \href{http://scikit-learn.org}{scikit-learn},
     a popular package for performing machine learning in Python.  I
     have contributed in many areas, but most notably routines for efficient
     2-point (e.g. nearest neighbors) queries, and algorithms based on these
     such as {\it k}-neighbor classification, kernel density estimation,
     and manifold learning.  I have also presented tutorials on the subject
     on many occasions, including at the PyCon, SciPy, and PyData
     conferences.}

   \subtopic[2011--Present]{SciPy}{
     I am a maintainer of 
     \href{http://scipy.org}{SciPy}, the definitive repository for many
     scientific computing tools available in Python.
     My contributions are primarily in the sparse
     matrix package, including code for efficient solutions of large sparse
     eigenvalue problems, and for efficient traversal and analysis of
     large sparse graphs.
   }

   \subtopic[2012--Present]{AstroML}{
     I am the primary author of \href{http://astroML.org}{AstroML}, a Python
     package devoted to Machine Learning in Astronomy and Astrophysics.
     Drawing from tools available in SciPy, Scikit-Learn, Matplotlib, and
     other packages, it provides additional astronomy-specific data analysis
     routines, loaders for open astronomical datasets, and over 200 examples
     of data mining, machine learning, and visualization in Astronomy.
   }

   \subtopic[2013--Present]{SciDB-Py}{
     I am the primary author of \href{http://jakevdp.github.io/SciDB-py}
     {SciDB-py}, a Python wrapper of the SciDB database system aimed at
     efficient distributed array-based computation.  This project is in
     conjunction with engineers at Paradigm4 and at ContinuumIO.
   }

   \subtopic[2013--Present]{mpld3}{
     I am the author of the \href{http://github.com/jakevdp/mpld3}
     {mpld3} package, a Python module which converts maplotlib images into
     interactive D3js visualizations suitable for web publication.
   }

   \subtopic[]{Et Cetera}{
     I have made contributions to many other Python projects, including
     Matplotlib, IPython, NumPy, Pelican, Hyde, and others.  I have also
     open-sourced much of my research code and teaching materials.
   }

\smalltopic{Service}{}

   \subtopic[2014]{SciPy}{
     Tutorial co-chair for SciPy 2014.
   }

   \subtopic[2014]{PyCon}{
     Member of the tutorial review committee for PyCon 2014.
   }

   \subtopic[2012-2013]{PyData}{
     Member of the talk \& tutorial review committee for several PyData
     conferences.
   }


\pagebreak
%%%%%%%%%%%%%%%%%%%%%%%%%%%%%%%%%%%%%%%%%%%%%%%%%%%%%%%%%%%%%%%%%%%%%%%%%%%
\topic{S \large\hskip -1ex ELECTED TALKS}{\small \\{\textcolor{gray}{\it $*$= invited talk}}}

\smalltopic{Astronomy}{}

  \dateonly[*November 2013]{
    {\it Information Theory and Survey Design}\\
    LBL Cosmology Seminar, Berkeley CA
  }

  \dateonly[October 2013]{
    {\it LSST and the Time-domain Universe}\\
    Calvin College Physics Seminar, Grand Rapids, MI
  }

  \dateonly[*October 2013]{
    {\it Unlocking the Universe with Python and LSST}\\
    RuPy conference, Budapest, Hungary
  }

  \dateonly[*August 2013]{
    {\it Reproducible Astronomy in the LSST Era}\\
    Data Science Seminar, Los Alamos National Labs
  }

  \dateonly[July 2013]{
    {\it Opening Up Astronomy with Python and AstroML}\\
    Jake Vanderplas, Andrew Connolly, \& Zeljko Ivezic\\
    Scipy 2013, Austin TX
  }

  \dateonly[*May 2013]{
    {\it Information Theory and Survey Design}\\
    UC Davis Cosmology Seminar, Davis CA
  }

  \dateonly[*April 2013]{
    {\it Observational Tracers of Modified Gravity: Dwarf Disk Galaxies}\\
    Novel Probes of Gravity Workshop, University of Pennsylvania
  }

  \dateonly[October 2012]{
    {\it AstroML: Machine Learning for Astronomy}\\
    Conference on Intelligent Data Understanding, Boulder CO
  }

  \dateonly[July 2012]{
    {\it AstroML: Machine Learning for Astronomy}\\
    SciPy Conference, Austin TX
  }
    
  \dateonly[December 2011]{
    {\it Processing Shear Maps with Karhunen-Loeve Analysis} (poster)\\
    Jake Vanderplas, Bhuvnesh Jain, \& Andrew Connolly\\
    Neuro-Imaging Processing Symposium (NIPS), Granada Spain
  }
    
  \dateonly[*October 2011]{
    {\it Alternatives to 2-Point Statistics in Weak Lensing}\\
    DES Collaboration meeting, Philadelphia PA
  }
    
  \dateonly[*June 2011]{
    {\it Digital Planetariums for the Masses}\\
    AstroVis, University of Washington
  }
    
  \dateonly[May 2011]{
    {\it KL Interpolation of Weak Lensing Shear}\\
    INPA Cosmology Seminar, Lawrence Berkeley National Laboratory, CA
  }
    
  \dateonly[May 2011]{
    {\it KL Interpolation of Weak Lensing Shear}\\
    UC Davis Cosmology Seminar, Davis CA
  }
    
  \dateonly[May 2011]{
    {\it KL Interpolation of Weak Lensing Shear}\\
    KIPAC Cosmology Seminar, SLAC National Laboratory, CA
  }
    
  \dateonly[January 2011]{
    {\it Finding the Odd One Out in Spectroscopic Surveys} (poster)\\
    A. Connolly, S. Daniel, L. Xiong, J. Vanderplas, \& J. Schneider\\
    217th AAS meeting, Seattle WA
  }

  \dateonly[January 2011]{
    {\it 3D Reconstruction of the Density Field} (poster)\\
    Jake Vanderplas \& Andrew Connolly\\
    217th AAS meeting, Seattle WA
  }
    
  \dateonly[July 2010]{
    {\it A New Approach to Tomographic Mapping}\\
    Ten Years of Cosmic Shear, Edinburgh, UK
  }

  \dateonly[*November 2007]{
    {\it SALT-2 Light-curve Fitting for SDSS Supernovae}\\
    SDSS Collaboration Meeting, Fermi National Accelerator Laboratory
  }



\pagebreak
\smalltopic{Computing Talks \& Tutorials}{}

  \dateonly[November 2013]{
    {\it Financial Time-series Data in SciDB}\\
    Bryan Lewis \& Jake Vanderplas\\
    PyData NYC 2013
    }
  \dateonly[November 2013]{
    {\it Efficient Computing with NumPy}\\
    PyData NYC 2013
    }
  \dateonly[November 2013]{
    {\it Machine Learning with Scikit-Learn}\\
    PyData NYC 2013
    }
  \dateonly[August 2013]{
    {\it Big Analytics for Python Users Without the Hassles}\\
    Jake Vanderplas, Bryan Lewis, \& Travis Oliphant\\
    Webinar presented by Paradigm4
  }
  \dateonly[*July 2013]{
    {\it Interactive Computing with IPython and ASCOT}\\
    Clawpack Workshop, University of Washington
  }
  \dateonly[July 2013]{
    {\it An Introduction to Scikit-Learn (2-part, 8-hour tutorial)}\\
    Jake Vanderplas, Gael Varoquaux, \& Olivier Grisel\\
    Scipy 2013, Ausin TX
  }
  \dateonly[July 2013]{
    {\it Introduction to Python (3-hour tutorial)}\\
    Software Carpentry Course, Seattle WA
  }
  \dateonly[April 2013]{
    {\it Interactive Applications with Matplotlib (2-hour tutorial)}\\
    PyData Silicon Valley, Santa Clara CA
  }
  \dateonly[April 2013]{
    {\it An Introduction to Scikit-Learn (3-hour tutorial)}\\
    PyCon 2013, Santa Clara CA
  }
  \dateonly[*October 2012]{
    {\it Scientific Machine Learning with Scikit-learn (1-hour tutorial)}\\
    {\it Interactive Visualization with Matplotlib (1-hour tutorial)}\\
    PyData NYC, New York NY
  }
  \dateonly[July 2012]{
    {\it Machine Learning in Python (4-hour tutorial)}\\
    Scipy 2012, Austin TX
  }
  \dateonly[*March 2012]{
    {\it Scikit-Learn Tutorial (1-hour tutorial)}\\
    PyData Workshop, Google Campus, Mountain View CA
  }



\smalltopic{General Audience}{}

  \dateonly[*June 2013]{
    {\it The Science of Time Travel}\\
    at the event {\it Short Films, Big Ideas: The Science of Science Fiction}\\
    Seattle International Film Festival, Seattle WA
  }

  \dateonly[*March 2012]{
    {\it Dark Matter, Dark Energy, and the Fate of the Universe}\\
    Calvin College Physics Colloquium, Grand Rapids MI
  }

  \dateonly[*November 2011]{
    {\it Kinect/WorldWide Telescope Demonstration}\\
    Supercomputing 2011, Seattle WA
  }

  \dateonly[*November 2011]{
    {\it WorldWide Telescope Demonstration}\\
    Partners in Learning Global Forum, Washington DC
  }

  \dateonly[*November 2011]{
    {\it Gravity: A Lens to the Universe}\\
    KCTS9 Queen Anne Science Cafe, Seattle WA
  }

  \dateonly[*October 2011]{
    {\it WorldWide Telescope Demonstration}\\
    Popular Mechanics Breakthrough Awards, New York NY
  }

  \dateonly[*March 2011]{
    {\it Understanding the Dark Side of the Universe}\\
    Pacific Science Center's ``Science with a Twist'', Seattle WA
  }

  \dateonly[*February 2011]{
    {\it Interconnection in Art and Cosmology}\\
    at the {\it Traces of the Universe} Art show,\\
    University of Washington, Seattle WA
  }

  \dateonly[May 2009]{
    {\it Dark Matter, Gravitational Lensing, and Cosmology}\\
    Battle Point Astronomical Society, Bainbridge Island, WA
  }



%%%%%%%%%%%%%%%%%%%%%%%%%%%%%%%%%%%%%%%%%%%%%%%%%%%%%%%%%%%%%%%%%%%%%%%%%%%
\topic{P \large\hskip -1ex UBLICATIONS}{~}
\subtopic{\hspace*{-3ex} Books}{~ 
  \begin{itemize}[leftmargin=0ex, itemsep=0ex, parsep=.5ex, labelindent=-4ex]

    \item[{\bf \textcolor{myblue}{[1]}}]
      Z. Ivezic, A. Connolly, J. Vanderplas \& A. Gray.
      {\sl Statistics, Data Mining and Machine Learning in Astronomy.}
      Princeton Univertiy Press, 2014
  \end{itemize}
}

\subtopic{\hspace*{-3ex} Articles}{~ 
  %\vspace*{\baselineskip}
  \begin{itemize}[leftmargin=0ex, itemsep=0ex, parsep=.5ex, labelindent=-4ex]

    \item[{\bf \textcolor{myblue}{[2]}}]
      Lars Buitinck {\it et al.}
      {\it API design for machine learning software:
        experiences from the scikit-learn project}
      European Conference on Machine Learning and Principles and Practices
      of Knowledge Discovery in Databases (2013)

    \item[{\bf \textcolor{myblue}{[4]}}]
      L. Palaversa {\sl et al.}
      {\sl Exploring the Variable Sky with LINEAR. III.
        Classification of Periodic Light Curves}
      AJ 146:101, 2013

    \item[{\bf \textcolor{myblue}{[3]}}]
      V. Vikram, A. Cabr\'{e}, B. Jain, \& J. Vanderplas.
      {\it Astrophysical tests of modified gravity:
        the morphology and kinematics of dwarf galaxies}
      JCAP 08:20, 2013
 
    \item[{\bf \textcolor{myblue}{[5]}}]
      J. Vanderplas, A. Connolly, Z. Ivezic, \& A. Gray.
      {\sl Introduction to AstroML: Machine Learning for Astrophysics}.
      Proc. of the CIDU, 2012
      {\bf (Recipient of the CIDU 2012 Best Paper award)}

    \item[{\bf \textcolor{myblue}{[6]}}]
      J. Vanderplas, A. Connolly, B. Jain, \& M. Jarvis.
      {\it Interpolating Masked Weak Lensing Signals with Karhunen-Loeve
        Analysis}.
      ApJ 744:180, 2012

    \item[{\bf \textcolor{myblue}{[7]}}]
      4. S. Daniel, A. Connolly, A.J. J. Schneider, J. Vanderplas, \& L. Xiong
      {\sl Classification of Stellar Spectra with LLE}.
      AJ 142:203, 2011

    \item[{\bf \textcolor{myblue}{[8]}}]
      5. F. Pedregosa {\sl et al.}
      {\sl Scikit-learn: Machine learning in Python}.
      Journal of Machine Learning Research, 12:2825, 2011

    \item[{\bf \textcolor{myblue}{[9]}}]
      B. Jain \& J. Vanderplas.
      {\sl Tests of Modified Gravity with Dwarf Galaxies}.
      JCAP 10:32, 2011

    \item[{\bf \textcolor{myblue}{[10]}}]
      J. Vanderplas, A. Connolly, B. Jain, \& M. Jarvis.
      {\sl 3D Reconstruction of the Density Field: An SVD Approach
        to Weak Lensing Tomography}.
      ApJ 727:118, 2011

    \item[{\bf \textcolor{myblue}{[11]}}]
      L. Xiong, B. Poczos, J. Schneider, A. Connolly, \& J. Vanderplas.
      {\sl Hierarchical Probabilistic Models for Group Anomaly Detection}.
      Artificial Intelligence and Statistics (AISTATS), 2011

    \item[{\bf \textcolor{myblue}{[12]}}]
      H. Lampeitl {\sl et al.}
      {\sl First-year Sloan Digital Sky Survey-II supernova results: 
      consistency and constraints with other intermediate-redshift data sets.}
      MNRAS 401:2331, 2010

    \item[{\bf \textcolor{myblue}{[13]}}]
      LSST Science Collaboration
      {\sl LSST Science Book, Version 2.0}, arXiv:0912.0201, 2010

    \item[{\bf \textcolor{myblue}{[14]}}]
      R. Kessler {\it et al.}
      {\it First-Year Sloan Digital Sky Survey-II Supernova Results: 
      Hubble Diagram and Cosmological Parameters}.
      ApJS 185:32, 2009

    \item[{\bf \textcolor{myblue}{[15]}}]
      J. Vanderplas \& A. Connolly.
      {\it Reducing the Dimensionality of Data: Locally 
      Linear Embedding of Sloan Galaxy Spectra}.
      AJ 138:1365, 2009

    \item[{\bf \textcolor{myblue}{[16]}}]
      J. Sollerman {\sl et al.}
      {\it First-Year Sloan Digital Sky Survey-II (SDSS-II) Supernova 
      Results: Constraints on Nonstandard Cosmological Models}.
      ApJ 703:1374, 2009

    \item[{\bf \textcolor{myblue}{[17]}}]
      R. Kessler {\it et al.}
      {\it SNANA: A Public Software Package for Supernova Analysis}.
      PASP 121:1028, 2009

  \end{itemize}
}



\end{document}
