% Adapted from layout by GaelVaroquaux
%  http://gael-varoquaux.info
%

\documentclass{article} %{{{--

\usepackage[paper=a4paper,
	    top=2cm,
	    left=1.35cm,
	    width=18.2cm,
	    bottom=2cm
	    ]{geometry}
            %margin=4cm,

\usepackage{calc}
\usepackage[T1]{fontenc}
\usepackage[utf8]{inputenc}
\usepackage{lmodern}
\usepackage{color,hyperref}
\usepackage{graphicx}
\usepackage{multicol}

\usepackage{wasysym} % For phone symbol
\usepackage{url}

\def\bf{\bfseries}
\def\sf{\sffamily}
\def\sl{\slshape}
% Semi condensed bold

\definecolor{deep_blue}{rgb}{0,.2,.5}
\definecolor{dark_blue}{rgb}{0,.1,.3}
\definecolor{myblue}{rgb}{.01,0.21,0.71}
\definecolor{gray}{rgb}{.5, .5, .5}

\hypersetup{pdftex,  % needed for pdflatex
  breaklinks=true,  % so long urls are correctly broken across lines
  colorlinks=true,
  urlcolor=myblue,
  %linkcolor=darkblue,
  %citecolor=darkgreen,
  }


%% This gives us fun enumeration environments. 
\usepackage{enumitem}

%% More layout: Get rid of indenting throughout entire document
\setlength{\parindent}{0in}

%% Reference the last page in the page number
%
\usepackage{fancyhdr,lastpage}
\pagestyle{fancy}
%\pagestyle{empty}      % Uncomment this to get rid of page numbers
\fancyhf{}\renewcommand{\headrulewidth}{0pt}
\fancyfootoffset{\marginparsep+\marginparwidth}
\lfoot{%\hspace{\footpageshift}%
	\,\hfill %
                    \arabic{page} of \protect\pageref*{LastPage} % +LP
%                    \arabic{page}                               % -LP
                    \hfill \,}

\newcommand{\mydate}[1]{{\textcolor{gray}{\footnotesize #1}}}


\newcommand{\makeheading}[1]%
        {%\hspace*{-\marginparsep minus \marginparwidth}%
         %\begin{minipage}[t]{\textwidth+\marginparwidth+\marginparsep}%
         \begin{minipage}[t]{\textwidth}%
                {\Large #1}\\%[-0.5\baselineskip]%
                 \color{deep_blue}{\rule{\columnwidth}{3pt}}%
         \end{minipage}
	 \vskip 1.\baselineskip plus 2\baselineskip minus 1.\baselineskip
	}

\newlength\sidebarwidth
\setlength\sidebarwidth{3.6cm}

\newcommand{\topic}[3][]%
	 {\pagebreak[2]%
	 \vskip 1.5\baselineskip plus 3\baselineskip minus 0.7\baselineskip
	 \begin{minipage}{\textwidth}
         \phantomsection\addcontentsline{toc}{section}{#1}%
         \nopagebreak\hspace{0in}%
         \nopagebreak\begin{minipage}[t]{\sidebarwidth - .2cm}
         \raggedleft \bf\sf 
	 \color{deep_blue}{\Large #2}
	 \end{minipage}%
	 \hfill
	 \begin{minipage}[t]{\linewidth - \sidebarwidth}
	 \nopagebreak{\color{deep_blue}%
		    \rule{0pt}{\baselineskip}%
		    \rule{\linewidth}{2.5pt}%
		    \llap{\raisebox{.3\baselineskip}{\sf #1}}%
		    \vspace*{.1\baselineskip}%
		    }%
	 #3%
	 \end{minipage}
	 \end{minipage}}

\newcommand{\smalltopic}[2]%
	 {\pagebreak[2]%
	 \vskip 1\baselineskip plus 2\baselineskip minus 0.3\baselineskip
	 \begin{minipage}{\textwidth}
	 %\hspace{-\marginparsep minus \marginparwidth}%
         \phantomsection\addcontentsline{toc}{subsection}{#1}%
         \nopagebreak\hspace{0in}%
         \nopagebreak\begin{minipage}[t]{\sidebarwidth - .2cm}
         \raggedleft \bf\sf %\vskip -0.5\baselineskip
	 \textcolor{dark_blue}{\large #1}%
	 \end{minipage}%
	 \hfill
	 \begin{minipage}[t]{\linewidth - \sidebarwidth}
	 \nopagebreak{%
	    %\vspace{-.7\baselineskip}%
	    \rule{\linewidth}{.5pt}%
	    \vspace{.1\baselineskip}%
	    }%
	    #2
	 \end{minipage}
	 \end{minipage}}

\newcommand{\subtopic}[3][]
	 {\begin{minipage}{\textwidth}
	 \vspace*{.4\baselineskip}
         \nopagebreak\hspace{0in}%
         \nopagebreak\begin{minipage}[t]{\sidebarwidth - .2cm}
	 % Super posh: using semi-bold condensed fonts. Works only with
	 % lmodern
         \raggedleft {\sf\fontseries{sbc}\selectfont #2}
	 {\small\sl\\[-0.2\baselineskip] #1}
	 \end{minipage}%
	 \hfill
	 \begin{minipage}[t]{\linewidth - \sidebarwidth}
	 #3%
	 \end{minipage}%
	 \vspace*{.2\baselineskip plus 1\baselineskip minus
	 .2\baselineskip}%
	 \end{minipage}}

\newcommand{\sidenote}[2]
	 {\vspace*{-.2\baselineskip}\begin{minipage}{\textwidth}
         \nopagebreak\hspace{0in}%
         \nopagebreak\begin{minipage}[t]{\sidebarwidth - .2cm}
         \raggedleft {#1}
	 \end{minipage}%
	 \hfill
	 \begin{minipage}[t]{\linewidth - \sidebarwidth}
	 #2%
	 \end{minipage}%
	 \vspace*{.5\baselineskip plus 1\baselineskip minus
	 .2\baselineskip}%
	 \end{minipage}}

% New lists environments 
\newlist{outerlist}{itemize}{1}
\setlist[outerlist]{font=\sffamily\bfseries, label=\textbullet}
\setitemize{topsep=0ex, partopsep=0ex}
\setdescription{font=\normalfont\sffamily\bfseries, itemsep=.5ex,
    parsep=.5ex, leftmargin=3ex}

\newcommand{\blankline}{\quad\pagebreak[2]}

\def\mydot{\textcolor{deep_blue}{\rule{1ex}{1ex}}}

%%%%%%%%%%%%%%%%%%%%%%%%%%%%%%%%%%%%%%%%%%%%%%%%%%%%%%%%%%%%%%%%%%%%%--}}}%
\begin{document}
\makeheading{
\begin{minipage}[B]{0.5\textwidth}
    %\vfill
    \vspace*{-.5\baselineskip}%
    \parbox{10cm}{
	\hskip -0.1cm
	{\Huge\bf\sf \color{deep_blue} Jake %
		\color{dark_blue} V\huge \hskip -0.05cm ANDERPLAS}
	\\[-.2\baselineskip]
	\bf\sf Astronomy -- Machine Learning
    }
\end{minipage}
\hfill
\begin{minipage}[B]{8cm}
    \raggedleft
    \,\vskip -1em
    \small
        University of Washington (Seattle, WA)
	{\texttt {jakevdp@cs.washington.edu}}%
    \vspace*{-.5\baselineskip}%
\end{minipage}
}

%\begin{center} 
%\begin{minipage}{15cm}
\begin{multicols}{2}
\sloppy

\textcolor{deep_blue}{\bf\sf Research interests}:
data mining and automated learning of astronomical data sets.

\begin{itemize}[leftmargin=2ex, itemsep=0ex]
\item[\mydot]
My PhD research focused on {\it weak lensing}, a technique utilizing small
gravitational perterbations of light paths to learn about the distribution
of matter in the universe.

\item[\mydot]
Currently I am an NSF post-doctoral fellow, working jointly between the
Astronomy and Computer Science departments at the University of Washington.

\item[\mydot]
Additionally, I am interested in developing open source scientific-computing
tools (mainly in the Python programming language) and encouraging open research
through these tools.

\end{itemize}
\end{multicols}
\vspace*{-1.5em}
\fussy
%\end{minipage}
%\end{center}

%%%%%%%%%%%%%%%%%%%%%%%%%%%%%%%%%%%%%%%%%%%%%%%%%%%%%%%%%%%%%%%%%%%%%%%%%%%
\topic{E \large\hskip -1ex DUCATION}{~}

    \subtopic[2006-2012]{\bf PhD}{
        University of Washington, Seattle, WA, advised by Andrew Connolly\\
	Thesis: \href{http://adsabs.harvard.edu/abs/2013arXiv1301.6657V}{
          Karhunen-Loeve Analysis for Weak Gravitational Lensing}
    }

    \subtopic[2006-2007]{\bf MS}{
      University of Washington, Seattle, WA, supervised by Craig Hogan\\
    }

    \subtopic[1999-2003]{\bf BS}{
      Calvin College, Grand Rapids MI\\
      Major: Physics; Minors: Mathematics \& Japanese
    }

%%%%%%%%%%%%%%%%%%%%%%%%%%%%%%%%%%%%%%%%%%%%%%%%%%%%%%%%%%%%%%%%%%%%%%%%%%%
\topic{E \large\hskip -1ex XPERIENCE}{~}

\vspace*{-0.5\baselineskip}
\smalltopic{Employment}{}

    \subtopic[2013--Present]{UW Computer Science}{
      NSF post-doctoral fellowship, CI-TraCS program.\\
      Department of Computer Science, University of Washington.
      Supervised by Magda Balazinska}

    \subtopic[2012--2013]{UW Astronomy}{
      Postdoctoral Research in the LSST Image Simulation group.\\
      Department of Astronomy, University of Washington.
      Supervised by Andrew Connolly
    }

    \subtopic[2010--2012]{UW Planetarium}{
      WorldWide Telescope Planetarium Project Coordinator\\
      University of Washington Planetarium, Seattle WA \& Microsoft Research, Redmond WA
    }

    \subtopic[2008--2010]{UW Planetarium}{
      K-12 and Community Outreach Coordinator\\
      University of Washington Planetarium, Seattle WA
    }

    \subtopic[2004--2006]{Mount Hermon}{
      Experiential Science Educator (4th-8th grade)\\
      Mount Hermon Outdoor Science School, Santa Cruz CA
    }

    \subtopic[2004--2005]{Summit Adventure}{
      Backpacking, Rock Climbing, and Mountaineering Instructor\\
      Summit Adventure, Bass Lake CA
    }

    \subtopic[2003--2004]{Japan ESL}{
      Teacher and Tutor of English as a second language,\\
      Sendai Gakusei Sentaa, Sendai, Japan
    }

\smalltopic{Volunteering}{}

    \subtopic[2010--2013]{Neighborhood Advocacy}{
      As a founder of West Seattle Greenways and transportation chair of
      the Delridge Neighborhood Council, I led in securing several grants
      totaling over \$2 million for pedestrian
      and bicycle safety improvements in the neighborhood.}

    \subtopic[2009--2013]{Pacific Science Center}{
      As a Science Communication Fellow, I facilitate activities for
      museum visitors and give occasional community talks on astronomy
      and astrophysics.
    }

    \subtopic[2007--2012]{Sierra Club}{
      As a program leader for the Sierra Club's {\it Inner City Outings}
      program, I led 3-4 hiking \& camping trips each year with inner-city
      youth.
    }

\smalltopic{Students Mentored}{}

    \subtopic[2008--2009]{Andy Barr \&\\Devon McMinn}{
      Undergraduate, University of Washington Pre-MAP\\
      {\it Astronomical Data Processing with LLE}
    }

    \subtopic[2012--2013]{SungWon Kwok}{
      Undergraduate, University of Washington Astronomy\\
      {\it Superimposed High Redshift Spectra}
    }

%%%%%%%%%%%%%%%%%%%%%%%%%%%%%%%%%%%%%%%%%%%%%%%%%%%%%%%%%%%%%%%%%%%%%%%%%%%
\topic{C \large\hskip -1ex OMPUTING}{
  I am an active developer and maintainer of scientific computing packages
  in the Python community.
  See my github profile (\href{http://github.com/jakevdp}{jakevdp})
  for details.}

%\vspace*{-0.5\baselineskip}
\smalltopic{Skills}{

  \begin{itemize}[leftmargin=0ex, itemsep=0ex, labelindent=-2ex, parsep=.5ex]
    \item[\mydot] Proficient developer, with a specialization in
      scientific computing, including data mining and machine learning.
    \item[\mydot] Expert in the Python Language and extensions such as Cython;
      very good knowledge of C, C++, and interfacing to legacy Fortran code.
    \item[\mydot] Author of a popular Python blog covering visualization,
      scientific computing, and fun distractions:
      \href{http://jakevdp.github.io}{Pythonic Perambulations}
  \end{itemize}

}

\smalltopic{Software}{}

   \subtopic[2010--Present]{Scikit-Learn}{
     I a member of the core team of
     \href{http://scikit-learn.org}{Scikit-Learn},
     a very popular package for machine learning in Python.  I
     have contributed in many areas, but most notably routines for efficient
     2-point (e.g. nearest neighbors) queries, and algorithms based on these
     such as {\it k}-neighbor classification, kernel density estimation,
     and manifold learning.  I have also presented tutorials on the subject
     on many occasions, includinng at the PyCon, SciPy, and PyData
     conferences.}

   \subtopic[2011--Present]{SciPy}{
     I am a maintainer of 
     \href{http://scipy.org}{SciPy}, the definitive repository for many
     scientific computing tools available in Python.
     My contributions are primarily in the sparse
     matrix package, including code for efficient solutions of large sparse
     eigenvalue problems, and for efficient traversal and analysis of
     large sparse graphs.
   }

   \subtopic[2012--Present]{AstroML}{
     I am the primary author of \href{http://astroML.org}{AstroML}, a Python
     package devoted to Machine Learning in Astronomy and Astrophysics.
     Drawing from tools available in SciPy, Scikit-Learn, Matplotlib, and
     other packages, it provides additional astronomy-specific data analysis
     routines, loaders for open astronomical datasets, and over 200 examples
     of data mining, machine learning, and visualization in Astronomy.
   }

   \subtopic[2013--Present]{SciDB-Py}{
     I am the primary author of \href{http://jakevdp.github.io/SciDB-py}
     {SciDB-py}, a Python wrapper of the SciDB database system aimed at
     efficient distributed array-based computation.  This project is in
     conjunction with engineers at Paradigm4 and at ContinuumIO.
   }

   \subtopic[2013--Present]{JSAnimation}{
     I am the author of the \href{http://github.com/jakevdp/JSAnimation}
     {JSAnimation} package, a lightweight tool to embed interactive
     matplotlib animations within html pages, including IPython notebooks.
   }

   \subtopic[]{Others}{
     I have made contributions to many other Python projects, the most
     noteworthy of which are matplotlib for visualization,
     ipython for interactive computing, pelican for static blog generation,
     and others.
   }

%%%%%%%%%%%%%%%%%%%%%%%%%%%%%%%%%%%%%%%%%%%%%%%%%%%%%%%%%%%%%%%%%%%%%%%%%%%
\topic{H \large\hskip -1ex ONORS}{~}

    \subtopic[2013]{Data Visualization}{
      Runner-up in the 2013 {\it John Hunter Excellence in Plotting Competition}
    }

    \subtopic[2012]{CIDU Best Paper}{
      Recipient of the Best Paper Award, 2012 Conference on Intelligent
      Data Understanding (CIDU).
    }

    \subtopic[2012]{NSF Fellowship}{
      Recipient of an NSF prize fellowship (3 years) through the
      {\it CyberInfrastructure and Transformative Computational Science}
      (CI-TraCS) program.  NSF Award \#1226371
    }


%%%%%%%%%%%%%%%%%%%%%%%%%%%%%%%%%%%%%%%%%%%%%%%%%%%%%%%%%%%%%%%%%%%%%%%%%%%
\topic{S \large\hskip -1ex ELECTED TALKS}{~}


\subtopic{\hspace*{-3ex}Astronomy and\\Astrophysics}{~
  \vspace*{\baselineskip}

  \begin{itemize}[leftmargin=0ex, itemsep=0ex, parsep=.5ex, labelindent=-4ex]

  % Berkeley Cosmology Seminar

  \item[\mydate{August 2013}]
    {\it Reproducible Astronomy in the LSST Era}\\
    Data Science Seminar, Los Alamos National Labs (invited)

  \item[\mydate{July 2013}]
    {\it Opening Up Astronomy with Python and AstroML}\\
    Scipy 2013, Austin TX

  \item[\mydate{May 2013}]
    {\it Information Theory and Survey Design}\\
    UC Davis Cosmology Seminar, Davis CA (invited)

  \item[\mydate{April 2013}]
    {\it Observational Tracers of Modified Gravity: Dwarf Disk Galaxies}\\
    Novel Probes of Gravity Workshop, University of Pennsylvania (invited)

  \item[\mydate{October 2012}]
    {\it AstroML: Machine Learning for Astronomy}\\
    Conference on Intelligent Data Understanding, Boulder CO\\
    {\bf Winner of the CIDU 2012 Best Paper Award}

  \item[\mydate{July 2012}]
    {\it AstroML: Machine Learning for Astronomy}\\
    SciPy Conference, Austin TX
    
  \item[\mydate{December 2011}]
    {\it Processing Shear Maps with Karhunen-Loeve Analysis} (poster)\\
    Neuro-Imaging Processing Symposium (NIPS), Granada Spain
    
  \item[\mydate{October 2011}]
    {\it Alternatives to 2-Point Statistics in Weak Lensing}\\
    DES Collaboration meeting, Philadelphia PA
    
  \item[\mydate{June 2011}]
    {\it Digital Planetariums for the Masses}\\
    AstroVis, University of Washington (invited)
    
  \item[\mydate{May 2011}]
    {\it KL Interpolation of Weak Lensing Shear}\\
    INPA Cosmology Seminar, Lawrence Berkeley National Laboratory, CA
    
  \item[\mydate{May 2011}]
    {\it KL Interpolation of Weak Lensing Shear}\\
    UC Davis Cosmology Seminar, Davis CA
    
  \item[\mydate{May 2011}]
    {\it KL Interpolation of Weak Lensing Shear}\\
    KIPAC Cosmology Seminar, SLAC National Laboratory, CA
    
  \item[\mydate{February 2011}]
    {\it Peak Statistics for Weak Lensing}\\
    University of Pennsylvania Lunch Talk, Philadelphia PA
    
  \item[\mydate{January 2011}]
    {\it Finding the Odd One Out in Spectroscopic Surveys} (poster)\\
    {\it 3D Reconstruction of the Density Field} (poster)\\
    217th AAS meeting, Seattle WA
    
  \item[\mydate{July 2010}]
    {\it A New Approach to Tomographic Mapping}\\
    Ten Years of Cosmic Shear, Edinburgh, UK

  \item[\mydate{November 2007}]
    {\it SALT-2 Light-curve Fitting for SDSS Supernovae}\\

  \end{itemize}
  }


\subtopic{\hspace*{-3ex}Python Talks\\ and Tutorials}{~ 
  \vspace*{\baselineskip}
  \begin{itemize}[leftmargin=0ex, itemsep=0ex, parsep=.5ex, labelindent=-4ex]

  % RuPy

  \item[\mydate{July 2013}]
    {\it Interactive Computing with IPython and ASCOT}\\
    Clawpack Workshop, University of Washington (invited)



  \end{itemize}
}


\subtopic{\hspace*{-3ex}Non-technical Talks}{~ 
  \begin{itemize}[leftmargin=0ex, itemsep=0ex, parsep=.5ex, labelindent=-4ex]


  \item[\mydate{June 2013}]
    {\it Einstein, Gravity, and Time Travel}\\
    Seattle International Film Festival ``Short Films, Big Ideas: Science Fiction'' (invited)

  \item[\mydate{March 2013}]
    {\it Dark Matter, Dark Energy, and the Fate of the Universe}\\
    Calvin College Physics Colloquium, Grand Rapids MI (invited)

  \item[\mydate{November 2011}]
    {\it Kinect/WorldWide Telescope Demonstration}\\
    Supercomputing 2011, Seattle WA (invited)

  \item[\mydate{November 2011}]
    {\it WorldWide Telescope Demonstration}\\
    Partners in Learning Global Forum, Washington DC (invited)

  \item[\mydate{November 2011}]
    {\it Gravity: A Lens to the Universe}\\
    KCTS9 Queen Anne Science Cafe, Seattle WA (invited)

  \item[\mydate{October 2011}]
    {\it WorldWide Telescope Demonstration}\\
    Popular Mechanics Breakthrough Awards, New York NY (invited)

  \item[\mydate{March 2011}]
    {\it Understanding the Dark Side of the Universe}\\
    Pacific Science Center's ``Science with a Twist'' (invited)

  \item[\mydate{May 2009}]
    {\it Dark Matter, Gravitational Lensing, and Cosmology}\\
    Battle Point Astronomical Society


  \end{itemize}
}



\end{document}
